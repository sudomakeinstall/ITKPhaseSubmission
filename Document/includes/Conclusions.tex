\section{Conclusions and Future Directions}

ITK is a large and powerful framework for medical (and general) image processing.  However, the lack of filters for understanding, manipulating, and unwrapping phase data is a current limitation of the library.  We have here presented an ITK module which we hope will begin to bridge that gap.  The two most significant contributions are the \code{itk::QualityGuidedPhaseUnwrappingImageFilter} and \code{itk::DCTPhaseUnwrappingImageFilter} classes, which implement efficient $n$-dimensional unwrapping algorithms.  The quality-guided approach has the advantage of giving a result that is congruent to the input.  Moreover, this approach avoids low-quality phase data, given an adequate quality map.  The unweighted $L^2$-norm approach has the advantage of giving a smooth result throughout, but is not congruent with the input and weights all pixels equally regardless of quality.

These algorithms both gave servicable results when presented with the SWI data, which had few residues within the region of interest, and in which the low quality data was largely relegated to the periphery.  However, both algorithms failed to produce an adequate result when presented with the more difficult HARP image.  In the case of the quality-guided approach, this is likely due to the inadequacy of phase derivative variance as a quality map, because in HARP images phase varies quite smoothly even in regions where there is little to no signal.  In the case of the DCT algorithm, this is likely because the region of interest is relatively small compared to the image as a whole.

In the future, it would be of great benefit to allow for other quality maps (such as maximum phase gradient, pseudocorrelation, and user-defined masks) in addition to phase derivative variance.  This would allow for finer control over the path the algorithm takes in the case of difficult cases such as HARP images.  Additionally, it would be of benefit to implement a weighted $L^2$-norm method, so that the DCT approach could also exclude low-quality or uninteresting regions.  Weighted $L^2$-norm phase unwrapping algorithms have been described which iteratively apply unweighted algorithms to weighted wrapped phase Laplacians.  The preconditioned conjugate gradient (PCG) approach in particular makes use of this method \cite{Ghiglia1998}, and would be an important next step in the development of this module.

This submission has also described \code{itk::DCTImageFilter} and \code{itk::DCTPoissonSolverImageFilter}, which are efficient implementations of general-purpose utilities important in image compression, gradient image editing, and phase unwrapping.  The former is a simple wrapper to the FFTW library, allowing for the discrete cosine transform to be integrated into an ITK pipeline.  The latter makes use of the DCT class to recover an image from its Laplacian.  We refer the interested reader to the appendices for a proper discussion.